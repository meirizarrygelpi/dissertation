\chapter{Gamma, Beta, Zeta, Etcetera\label{AppGamma}}
%%%%%%%%%%%%%%%%%%%%%%%%%%%%%%%%%%%%%%%%%%%%%%%%%%%%%%%%%%%%%%%%%%%%%%%%%%%%%%%%%%%%%%%%%
In this appendix we list some properties of the Euler Gamma function, the Riemann zeta function and the binomial coefficient.
%%%%%%%%%%%%%%%%%%%%%%%%%%%%%%%%%%%%%%%%%%%%%%%%%%%%%%%%%%%%%%%%%%%%%%%%%%%%%%%%%%%%%%%%%
\section{Riemann zeta}
%%%%%%%%%%%%%%%%%%%%%%%%%%%%%%%%%%%%%%%%%%%%%%%%%%%%%%%%%%%%%%%%%%%%%%%%%%%%%%%%%%%%%%%%%
The Riemann zeta function $\zeta(s)$ is traditionally defined as an infinite series
\begin{equation}
	\zeta(s) \equiv \sum_{n = 1}^{\infty} \frac{1}{n^{s}}
\end{equation}
This function also appears for special values of the polylogarithm $\operatorname{Li}_{n}{(z)}$
\begin{equation}
	\operatorname{Li}_{k}{(z)} \equiv \sum_{n = 1}^{\infty} \frac{z^{n}}{n^{k}}
\end{equation}
That is,
\begin{equation}
	\zeta(s) = \operatorname{Li}_{s}{(1)}
\end{equation}
Some special values are
\begin{equation}
	\zeta(0) = - \frac{1}{2} \qquad \zeta(2) = \frac{\pi^{2}}{6} \qquad \zeta(4) = \frac{\pi^{4}}{90} \qquad \zeta(6) = \frac{\pi^{6}}{945}
\end{equation}
%%%%%%%%%%%%%%%%%%%%%%%%%%%%%%%%%%%%%%%%%%%%%%%%%%%%%%%%%%%%%%%%%%%%%%%%%%%%%%%%%%%%%%%%%
\section{Euler Gamma}
%%%%%%%%%%%%%%%%%%%%%%%%%%%%%%%%%%%%%%%%%%%%%%%%%%%%%%%%%%%%%%%%%%%%%%%%%%%%%%%%%%%%%%%%%
The Euler Gamma function $\Gamma(z)$ can be defined as an integral
\begin{equation}
	\Gamma(z) \equiv \int\limits_{0}^{\infty} \mathrm{d}\omega \left( \frac{1}{\omega} \right)^{1-z} \exp{(- \omega)}
\end{equation}
This function has singularities for $z = 0$ and $z = -n$ with $n$ a positive integer. As a sum of simple poles:
\begin{equation}
	\Gamma(z) = \sum_{n = 0}^{\infty} \frac{(-1)^{n}}{\Gamma(n + 1)(n + z)} + \int\limits_{1}^{\infty} \mathrm{d}T \left( \frac{1}{T} \right)^{1 - z} \exp{\left( - T \right)}
\end{equation}
One can write $\Gamma(z)$ as an infinite product
\begin{equation}
	\Gamma(z) = \frac{1}{z} \prod_{n = 1}^{\infty}\left( 1 + \frac{1}{n} \right)^{z} \left( 1 + \frac{z}{n} \right)^{-1}
\end{equation}
or equivalently
\begin{equation}
	\Gamma(z) = \frac{1}{z} \exp{\left[ z \sum_{n = 1}^{\infty} \log{\left( 1 + \frac{1}{n} \right)} - \sum_{k = 1}^{\infty} \log{\left( 1 + \frac{z}{k} \right)} \right]}
\end{equation}
So then, the reciprocal of $\Gamma(z)$ gives
\begin{equation}
	\frac{1}{\Gamma(z)} = z \exp{\left[ -z \sum_{n = 1}^{\infty} \log{\left( 1 + \frac{1}{n} \right)} + \sum_{k = 1}^{\infty} \log{\left( 1 + \frac{z}{k} \right)} \right]}
\end{equation}
Thus
\begin{equation}
\begin{split}
	\frac{\Gamma(z)}{\Gamma(w)} &= \frac{w}{z} \exp{\left[ (z - w) \sum_{n = 1}^{\infty} \log{\left( 1 + \frac{1}{n} \right)} - \sum_{k = 1}^{\infty} \log{\left( \frac{k + z}{k + w} \right)} \right]} \\
	&= \exp{\left[ (z - w) \sum_{n = 1}^{\infty} \log{\left( 1 + \frac{1}{n} \right)} - \sum_{k = 0}^{\infty} \log{\left( \frac{k + z}{k + w} \right)} \right]}
\end{split}
\end{equation}
In particular,
\begin{equation}
	\frac{\Gamma(1 - z)}{\Gamma(1 + z)} = \exp{\left[ - 2 z \sum_{n = 1}^{\infty} \log{\left( 1 + \frac{1}{n} \right)} - \sum_{k = 0}^{\infty} \log{\left( \frac{k + 1 - z}{k + 1 + z} \right)} \right]}
\end{equation}

Other identities are
\begin{equation}
	\frac{\Gamma(z + 1)}{\Gamma(z - 1)} = z (z - 1) \qquad \frac{\Gamma(- z  +1)}{\Gamma(- z - 1)} = z (z + 1)
\end{equation}
and
\begin{equation}
	\Gamma\left(z + \frac{1}{2} \right)  = \left(z - \frac{1}{2} \right) \Gamma\left(z - \frac{1}{2} \right) 
\end{equation}
More general identities are
\begin{equation}
\begin{split}
	\frac{\Gamma(z)}{\Gamma(z - n)} &= \prod_{k = 1}^{n} (z - k), \qquad n = 1, 2, 3, \ldots \\
	\frac{\Gamma(z + n)}{\Gamma(z)} &= \prod_{k = 1}^{n} (z + k - 1), \qquad n = 1, 2, 3, \ldots \\
	\frac{\Gamma(z + n)}{\Gamma(z - n)} &= \prod_{k = 1}^{n} (z + k - 1)(z - k), \qquad n = 1, 2, 3, \ldots
\end{split}
\end{equation}
and thus
\begin{equation}
\begin{split}
	\frac{\Gamma(-z)}{\Gamma(-z - n)} &= (-1)^{n} \prod_{k = 1}^{n} (z + k), \qquad n = 1, 2, 3, \ldots \\
	\frac{\Gamma(-z + n)}{\Gamma(-z)} &= (-1)^{n} \prod_{k = 1}^{n} (z - k + 1), \qquad n = 1, 2, 3, \ldots \\
	\frac{\Gamma(-z + n)}{\Gamma(-z - n)} &= \prod_{k = 1}^{n} (z - k + 1)(z + k), \qquad n = 1, 2, 3, \ldots
\end{split}
\end{equation}
Hence
\begin{equation}
	\frac{\Gamma(z)}{\Gamma(z - n)} = (-1)^{n} \frac{\Gamma(1 - z + n)}{\Gamma(1 - z)}, \qquad n = 0, 1, 2, \ldots
\end{equation}
or equivalently,
\begin{equation}
	\frac{\Gamma(z + 1)}{\Gamma(z + 1 - n)} = (-1)^{n} \frac{\Gamma(n - z)}{\Gamma(-z)}, \qquad n = 0, 1, 2, \ldots
\end{equation}

The Euler-Mascheroni constant $\gamma$ can be written as the difference of two divergent series
\begin{equation}
	\gamma = \sum_{k = 1}^{\infty} \left[ \frac{1}{k} - \log{\left( 1 + \frac{1}{k} \right)} \right]
\end{equation}
One also has another expression for $\gamma$ involving $\zeta(s)$
\begin{equation}
	\gamma = \sum_{k = 2}^{\infty} \frac{(-1)^{k} \zeta(k)}{k}
\end{equation}
which is a special case of
\begin{equation}
	\log{\left[ \Gamma(z + 1) \right]} + \gamma z = \sum_{k = 2}^{\infty} \frac{(-z)^{k} \zeta(k)}{k}
\end{equation}
and thus
\begin{equation}
	\frac{1}{\Gamma(z)} = z \exp{\left[ \gamma z - \sum_{k = 2}^{\infty} \frac{(-z)^{k} \zeta(k)}{k} \right]}
\end{equation}
The Euler reflection formula,
\begin{equation}
	\Gamma(z) \Gamma(1 - z) = \frac{\pi}{\sin{(\pi z)}}
\end{equation}
and Euler's product for sine,
\begin{equation}
	\frac{\sin{(\pi z)}}{\pi z} = \prod_{n = 1}^{\infty} \left(1 - \frac{z^{2}}{n^{2}} \right)
\end{equation}
lead to
\begin{equation}
	\frac{1}{\Gamma(1 + z) \Gamma(1 - z)} =\prod_{n = 1}^{\infty} \left(1 - \frac{z^{2}}{n^{2}} \right)
\end{equation}
A result from Gauss is
\begin{equation}
	\Gamma(n z) = \sqrt{\frac{2 \pi}{n}} \left( \frac{n^{z}}{\sqrt{2 \pi}} \right)^{n} \prod_{k = 1}^{n} \Gamma\left(z + \frac{n - k}{n} \right), \qquad n = 1, 2, 3, \ldots
\end{equation}
which leads to
\begin{equation}
	\Gamma(1 + n z) = z\sqrt{2 \pi n} \left( \frac{n^{z}}{\sqrt{2 \pi}} \right)^{n} \prod_{k = 1}^{n} \Gamma\left(z + \frac{n - k}{n} \right), \qquad n = 1, 2, 3, \ldots
\end{equation}

We can use the Euler Gamma to write the Riemann zeta as an integral. Using
\begin{equation}
	\left( \frac{1}{\kappa} \right)^{z} = \frac{1}{\Gamma(z)} \int\limits_{0}^{\infty} \mathrm{d}T \left( \frac{1}{T} \right)^{(1-z)} \exp{(- \kappa T)}
\end{equation}
we find
\begin{equation}
\begin{split}
	\zeta(s) &= \frac{1}{\Gamma(s)} \sum_{n = 1}^{\infty} \int\limits_{0}^{\infty} \mathrm{d}T \left( \frac{1}{T} \right)^{(1-s)} \exp{(- n T)} \\
	&= \frac{1}{\Gamma(s)} \int\limits_{0}^{\infty} \mathrm{d}T \left( \frac{1}{T} \right)^{(1-s)} \frac{\exp{(-T)}}{1 - \exp{(-T)}}
\end{split}
\end{equation}
%%%%%%%%%%%%%%%%%%%%%%%%%%%%%%%%%%%%%%%%%%%%%%%%%%%%%%%%%%%%%%%%%%%%%%%%%%%%%%%%%%%%%%%%%
\section{Euler Beta}
%%%%%%%%%%%%%%%%%%%%%%%%%%%%%%%%%%%%%%%%%%%%%%%%%%%%%%%%%%%%%%%%%%%%%%%%%%%%%%%%%%%%%%%%%
The Euler Beta function $B(x, y)$ can be written in terms of the Euler Gamma function:
\begin{equation}
	B(x, y) = \frac{\Gamma(x) \Gamma(y)}{\Gamma(x + y)}
\end{equation}
This form makes it manifest that $B(x, y) = B(y, x)$. There are two useful integral representations. The first is
\begin{equation}
	B(x, y) = \int\limits_{0}^{1} \mathrm{d}u \left( \frac{1}{u} \right)^{1-x} \left( \frac{1}{1 - u} \right)^{1 - y}
\end{equation}
and the second is
\begin{equation}
	B(x, y) = \int\limits_{0}^{\infty} \mathrm{d}T \left( \frac{1}{T} \right)^{1-x} \left( \frac{1}{1 + T} \right)^{x + y}
\end{equation}

We can use the Euler Beta to write
\begin{equation}
	\Gamma(x + y) = \frac{\Gamma(x) \Gamma(y)}{B(x, y)} \label{GammaPlus}
\end{equation}
Taking the limit $y \rightarrow x$ yields
\begin{equation}
	\Gamma(2x) = \frac{\Gamma^{2}(x)}{B(x, x)}
\end{equation}
This identity relates $\Gamma(2x)$ to $\Gamma^{2}(x)$. We can use this to write
\begin{equation}
	\Gamma(3x) = \frac{\Gamma(x) \Gamma(2x)}{B(x, 2x)} = \frac{\Gamma^{3}(x)}{B(x, x) B(x, 2x)}
\end{equation}
For any integer $n > 1$ we have
\begin{equation}
	\Gamma(n x) = \Gamma^{n}(x) \prod_{k = 1}^{n - 1} \frac{1}{B(x, k x)}
\end{equation}
This identity allows us to write the ratio of $\Gamma^{n}(x)$ and $\Gamma(n x)$ in terms of Euler Beta functions:
\begin{equation}
	\frac{\Gamma^{n}(x)}{\Gamma(n x)} = \prod_{k = 1}^{n - 1} B(x, k x)
\end{equation}

Using $x = x_{1}$ and $y = x_{2} + x_{3}$ in (\ref{GammaPlus}) gives
\begin{equation}
	\Gamma(x_{1} + x_{2} + x_{3}) = \frac{\Gamma(x_{1}) \Gamma(x_{2} + x_{3})}{B(x_{1}, x_{2} + x_{3})} = \frac{\Gamma(x_{1}) \Gamma(x_{2}) \Gamma(x_{3})}{B(x_{1}, x_{2} + x_{3}) B(x_{2}, x_{3})}
\end{equation}
Equivalently, we could have used $x = x_{2}$ and $y = x_{1} + x_{3}$ to get
\begin{equation}
	\Gamma(x_{1} + x_{2} + x_{3}) = \frac{\Gamma(x_{2}) \Gamma(x_{1} + x_{3})}{B(x_{2}, x_{1} + x_{3})} = \frac{\Gamma(x_{1}) \Gamma(x_{2}) \Gamma(x_{3})}{B(x_{2}, x_{1} + x_{3}) B(x_{1}, x_{3})}
\end{equation}
There is still a third possibility: use $x = x_{3}$ and $y = x_{1} + x_{2}$ to get
\begin{equation}
	\Gamma(x_{1} + x_{2} + x_{3}) = \frac{\Gamma(x_{3}) \Gamma(x_{1} + x_{2})}{B(x_{3}, x_{1} + x_{2})} = \frac{\Gamma(x_{1}) \Gamma(x_{2}) \Gamma(x_{3})}{B(x_{3}, x_{1} + x_{2}) B(x_{1}, x_{2})}
\end{equation}
Thus, we find the relation
\begin{equation}
\begin{split}
	B(x_{1}, x_{2} + x_{3}) B(x_{2}, x_{3}) &= B(x_{2}, x_{1} + x_{3}) B(x_{1}, x_{3}) \\
	&= B(x_{3}, x_{1} + x_{2}) B(x_{1}, x_{2})
\end{split}
\end{equation}
%%%%%%%%%%%%%%%%%%%%%%%%%%%%%%%%%%%%%%%%%%%%%%%%%%%%%%%%%%%%%%%%%%%%%%%%%%%%%%%%%%%%%%%%%
\section{Binomial}
%%%%%%%%%%%%%%%%%%%%%%%%%%%%%%%%%%%%%%%%%%%%%%%%%%%%%%%%%%%%%%%%%%%%%%%%%%%%%%%%%%%%%%%%%
Recall the binomial theorem,
\begin{equation}
	(1 + x)^{n} = \sum_{k = 0}^{n} \frac{1}{\Gamma(k+1)} \frac{\Gamma(n + 1)}{\Gamma(n - k + 1)} x^{k}
\end{equation}
This generalizes to any value $z$
\begin{equation}
	(1 + x)^{z} = \sum_{k = 0}^{\infty} \frac{1}{\Gamma(k+1)} \frac{\Gamma(z + 1)}{\Gamma(z - k + 1)} x^{k}
\end{equation}
Using the identity
\begin{equation}
	\frac{\Gamma(z + 1)}{\Gamma(z - k + 1)} = (-1)^{k} \frac{\Gamma(k - z)}{\Gamma(-z)}
\end{equation}
allows us to write
\begin{equation}
	\left( \frac{1}{1 - x} \right)^{z} = \sum_{k = 0}^{\infty} \frac{1}{\Gamma(k + 1)} \frac{\Gamma(z + k)}{\Gamma(z)} x^{k}
\end{equation}

In general, we have
\begin{equation}
	(a + b)^{n} = \sum_{k = 0}^{n} \frac{1}{\Gamma(k + 1)} \frac{\Gamma(n + 1)}{\Gamma(n - k + 1)} a^{k} b^{n - k}
\end{equation}
Then it follows that
\begin{align}
	(a + b)^{n} - a^{n} &= \sum_{k = 0}^{n - 1} \frac{1}{\Gamma(k + 1)} \frac{\Gamma(n + 1)}{\Gamma(n - k + 1)} a^{k} b^{n - k} \\
	(a + b)^{n} - b^{n} &= \sum_{k = 1}^{n} \frac{1}{\Gamma(k + 1)} \frac{\Gamma(n + 1)}{\Gamma(n - k + 1)} a^{k} b^{n - k} \\
	(a + b)^{n} - a^{n} - b^{n} &= \sum_{k = 1}^{n - 1} \frac{1}{\Gamma(k + 1)} \frac{\Gamma(n + 1)}{\Gamma(n - k + 1)} a^{k} b^{n - k}
\end{align}
We can use the binomial expansion recursively:
\begin{align}
	(a + b + c)^{n} &= \sum_{k_{1} = 0}^{n} \frac{1}{\Gamma(k_{1} + 1)} \frac{\Gamma(n + 1)}{\Gamma(n - k_{1} + 1)} a^{k_{1}} (b + c)^{n - k_{1}} \nonumber \\
	&= \sum_{k_{1} = 0}^{n} \sum_{k_{2} = 0}^{n - k_{1}} \frac{\Gamma(n + 1)}{\Gamma(k_{1} + 1) \Gamma(k_{2} + 1)} \frac{a^{k_{1}} b^{k_{2}} c^{n - k_{1} - k_{2}}}{\Gamma(n - k_{1} - k_{2} + 1)}
\end{align}
Then it follows that
\begin{equation}
	(a + b + c)^{n} - a^{n} - b^{n} - c^{n}
\end{equation}
can be written as
\begin{equation}
\begin{split}
	{}& \sum_{k_{1} = 1}^{n - 1} \sum_{k_{2} = 0}^{n - k_{1}} \frac{\Gamma(n + 1)}{\Gamma(k_{1} + 1) \Gamma(k_{2} + 1)} \frac{a^{k_{1}} b^{k_{2}} c^{n - k_{1} - k_{2}}}{\Gamma(n - k_{1} - k_{2} + 1)} \\
	&+ \sum_{k_{3} = 1}^{n - 1} \frac{1}{\Gamma(k_{3} + 1)} \frac{\Gamma(n + 1)}{\Gamma(n - k_{3} + 1)} b^{k_{3}} c^{n - k_{3}}
\end{split}
\end{equation}