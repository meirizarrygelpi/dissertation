\chapter{Dissertation Outline}
%%%%%%%%%%%%%%%%%%%%%%%%%%%%%%%%%%%%%%%%%%%%%%%%%%%%%%%%%%%%%%%%%%%%%%%%%%%%%%%%%%%%%%%%%
The main results of this work are scattering amplitudes for elastic events obtained after using the relativistic eikonal JWKB approximation. These results are contained in chapter \ref{Ch4Sca}. We consider scattering due to scalar, electromagnetic and gravitational interactions.

Before we get there, first we briefly review nonrelativistic quantum mechanics, the semiclassical (JWKB) approximation and the semiclassical eikonal approximation, which as we shall see in chapter \ref{ChNRPaths}, is a special case of the semiclassical approximation.

After reviewing the nonrelativistic theory, we present an example calculation in chapter \ref{ChCoulomb} where we compute a four-point scattering amplitude for matter particles interacting via the exchange of an instantaneous scalar wave. This system is equivalent to the problem of Coulomb scattering. Our result for the amplitude will exhibit an infinite set of singularities that correspond to two-body bound states. The steps we follow will generalize later to the relativistic theory. Indeed, this calculation will serve as a preview of the difficulties that we will encounter in the relativistic calculation.

Then we will boost into the relativistic theory. In chapter \ref{ChParticles} we begin by reviewing the action functionals for relativistic particles, and discuss the couplings of particles to different kinds of fields. We will also discuss the difference between coupling fields to other fields, and coupling fields to particles.

After this, in chapter \ref{ChRelaPaths}, we introduce the relativistic analogs of the concepts reviewed in chapter \ref{ChNRPaths}. This chapter introduces all the ingredients that are used in chapter \ref{Ch4Sca}.

There are many different directions along which our work can be continued further. We discuss some of these in chapter \ref{ChOut}.

We also include many appendices. In appendix \ref{AppFreeSca} we discuss the quantum and semiclassical kernels for free scalar particles. We discuss both massive and massless cases. Although free particles are not very interesting, this discussion will allow us to understand some of the differences between the quantum and the semiclassical kernels for massive particles. The calculations in this appendix are a good test of the validity of the tools introduced in chapter \ref{ChRelaPaths}.

In appendix \ref{AppMomInva} we review the different momentum invariants that can be constructed with the external momentum vectors. This discussion is brief and general, but in appendix \ref{AppKin4} we present a more detailed discussion relevant to the case with four external states (and specific to elastic scattering). In that appendix we also discuss different kinematical regimes, including the Regge limit and small-angle scattering.

Appendices \ref{AppGamma} and \ref{AppFourier} consist of a collection of well-known results involving the Euler Gamma and Beta functions, the Riemann zeta function, binomial combinatorics and Fourier transforms in arbitrary dimensions. Some of these results are used in the main body of work, but most are included for fun.

But before any of this, we begin with a historical overview in chapter \ref{ChHist}.